\documentclass[letterpaper,12pt]{article}

\usepackage[top=1in, bottom=1in, left=0.75in, right=0.75in]{geometry}
\usepackage{amssymb,amsmath,amsthm}
\usepackage{fancyhdr}
\usepackage{lastpage}
\usepackage{listings}
\usepackage[pdftex,
            pdfauthor={Nathaniel R. Reindl},
            pdftitle={CS 312, Spring 2009: Some Notes on Floating
              Point Arithmetic},
            pdfsubject={},
            pdfkeywords={},
            pdfproducer={},
            pdfcreator={}]{hyperref}

\setlength{\headheight}{0.25in}
\pagestyle{fancy}

\newtheorem{thm}{Theorem}[section]
\newtheorem{cor}[thm]{Corollary}
\newtheorem{lem}[thm]{Lemma}

\theoremstyle{remark}
\newtheorem{rem}[thm]{Remark}
\newtheorem{ans}[thm]{Answer}
\newtheorem*{ansa}{Answer}
\newtheorem*{sola}{Solution}

\theoremstyle{definition}
\newtheorem{defn}[thm]{Definition}
\newtheorem{que}[thm]{Question}
\newtheorem*{quea}{Question}
\newtheorem*{exma}{Example}

\lhead{\textbf{CS 312, Spring 2009}: {\it Some Notes on Floating Point
    Arithmetic.}}
\rhead{{\tt nrr@corvidae.org}}
\rfoot{Page \thepage{} of \pageref{LastPage}} \cfoot{}

\title{Some Notes on Floating Point}
\author{Nathaniel R. Reindl}
\date{\today}

\begin{document}

\section{Introduction}

Blah, blah, say some words to introduce the topic, maybe provide an
example, whatever.  On second thought, let's just dive head-first into
things and see where the current takes us.

\section{Normalized Real Numbers in Decimal}

Before we hit the meat of the topic too terribly hard, let's briefly
review scientific notation, significant places, and the process for
normalizing real numbers in the base that we're familiar with: base
10.  Additionally, we say real number to mean any number $x$ in
$\mathbb{R}$, the set of real numbers.

Scientific notation is a way of representing a real number that would
otherwise be cumbersome to be written conventionally.  We've all seen
this several thousand times already, so I'll save a lot of the detail
except for the form of the representation and what it means for a real
number to be normalized.

That said, a normalized real number $x$ in scientific notation has the
form
\begin{equation}
  x = a \cdot 10^b,
\end{equation}
where $1\leq \left | a \right | < 10$.  We call the coefficient $a$
one of either the {\it significand}, {\it mantissa}, or {\it fraction}
and $b$ the {\it characteristic} or {\it exponent}.  In 312, we used
the terms {\it fraction} and {\it exponent}.

The process of normalization itself is braindead simple to the point
that we can probably do it in our sleep.  Informally, we just shift
the decimal place as appropriate to ensure $1\leq \left | a \right | <
10$ and keep count of how many places we shift it.  The count of
places is, hence, $b$.

\section{Normalized Real Numbers in Some Base $\beta$}

Now that we've hit the material with which we're all familiar, we can
finally abstract the select few parts so that we can glean more
insight as to how this mess works as a whole.

Really, the changes to the definition we discussed earlier are
minimal.  A normalized real number $x$ in a base $\beta$ in
floating-point notation has the form
\begin{equation}
  x_\beta= a \cdot \beta^b,
\end{equation}
where $1\leq \left | a \right | < \beta$.  The same vocabulary
applies, and so on.  The process of normalization is even the same.

\section{Normalized Real Numbers in IEEE 754 Format}

It may seem like we're beating this normalization bit pretty heavily,
but we intend to say a few different words in this section compared to
the others.  Namely, we intend to introduce the mathematical
representation and some of the associated analysis formulas for a real
number represented using IEEE 754, and we intend to explain some of
the features of this format.  For specific examples and calculations,
we assume that we're dealing with double-precision floating-point
numbers.

Having said that, a real number $x$ in IEEE 754 floating-point format
(C/C++ {\tt float} or {\tt double}) has the form
\begin{equation}
  \label{ieee754_form}
  x=(-1)^s \cdot 2^{c-b}\cdot (1+m),
\end{equation}
where $s$ ($l_s=1$ bit) is an integer that represents the sign of $x$,
$c$ ($l_c=11$ bits) is the exponent obtained by normalizing $x$,
$b=1023$ is the bias used to determine signedness of the exponent
given $c$, and $m$ ($l_m=52$ bits) is the fraction obtained by
normalizing $x$.

\begin{quea}
  Why do we multiply $(1 + m)$ by $2^{c-b}$ and not just $m$?
\end{quea}
\begin{ansa}
  Since we're dealing with a binary representation of our number, we
  consider $\beta = 2$.  This means then that, since $1\leq \left | a
  \right | < \beta$, we always have $\left | a \right | = 1$.  Hence,
  our number $x_2$ in normalized form will always be
  \begin{equation*}
    x_2=1.m^{(1)}m^{(2)}\ldots m^{(l_m)}\cdot 2^b,
  \end{equation*}
  where each $m^{(i)}$ represents an individual bit.  The biggest
  consequence of this happens to be the fact that we don't store the
  extra bit of precision.  In essence, we get 53 bits of precision for
  the price of 52.
\end{ansa}

We computed the bias $b=1023$ by noting that the length of the
bitfield for our exponent is $l_c=11$ bits and then performing
\begin{equation}
  b = \left \lfloor \dfrac{2^{l_c}-1}{2} \right \rfloor 
  = \left \lfloor \dfrac{2^{11}-1}{2} \right \rfloor
  = \left \lfloor \dfrac{2047}{2} \right \rfloor
  = 1023.
\end{equation}
From here, we can compute $e_{\mathrm{min}}$ by
\begin{equation}
  \label{e_min}
  e_{\mathrm{min}} = b - 2^{l_c} - 2 = -1022
\end{equation}
and $e_{\mathrm{max}}$ by
\begin{equation}
  \label{e_max}
  e_{\mathrm{max}} = 2^{l_c} - b - 2 = 1023.
\end{equation}
\begin{quea}
  OK, wait.  Why isn't it the case that $e_{\mathrm{min}}=-1023$ and
  $e_{\mathrm{max}}=1024$?  Why are~\eqref{e_min} and~\eqref{e_max}
  both off by one?
\end{quea}
\begin{ansa}
  The IEEE 754 floating-point representation uses $e_{\mathrm{min}}-1$
  and $e_{\mathrm{max}}+1$ as special values to encode non-numeric
  results like NaNs, $\pm\infty$, and denormalized numbers.  We'll
  discuss the specifics of these later.
\end{ansa}
Going further, we can compute the minimum positive fraction
$m_{\mathrm{min}}$ by
\begin{equation}
  m_{\mathrm{min}}=1\cdot 2^{-l_m}=1\cdot
  2^{-52}=\dfrac{1}{4\,503\,599\,627\,370\,496}
\end{equation}
and the maximum positive fraction $m_{\mathrm{max}}$ by
\begin{equation}
  m_{\mathrm{max}}=\sum_{i=1}^{l_m}1\cdot=\sum_{i=1}^{52}1\cdot
  2^{-i}=\dfrac{4\,503\,599\,627\,370\,495}{4\,503\,599\,627\,370\,496}
\end{equation}
Now that we have $e_{\mathrm{min}}$, $e_{\mathrm{max}}$,
$m_{\mathrm{min}}$, $m_{\mathrm{max}}$, we can find the values for
$x_{\mathrm{min}}$ and $x_{\mathrm{max}}$.

To do this, we refer back to~\eqref{ieee754_form} and plug in values
as appropriate.  For $x_{\mathrm{min}}$, we let $s=0$,
$c-b=e_{\mathrm{min}}$ and $m=m_{\mathrm{min}}$.  We obtain then
\begin{align}
  x_{\mathrm{min}} &= (-1)^0\cdot 2^{e_{\mathrm{min}}}\cdot (1 + m_{\mathrm{min}})\\
  &= (-1)^0\cdot 2^{-1022}\cdot (1 + \dfrac{1}{4\,503\,599\,627\,370\,496})\\
  &= 2.22507\cdot 10^{-308}.
\end{align}
Similarly, for $x_{\mathrm{max}}$, we let $s=0$,
$c-b=e_{\mathrm{max}}$, and $m=m_{\mathrm{max}}$.  Hence,
\begin{align}
  x_{\mathrm{max}} &=(-1)^0\cdot 2^{e_{\mathrm{max}}}\cdot (1 + m_{\mathrm{max}})\\
  &= (-1)^0\cdot 2^{1023}\cdot (1+
  \dfrac{4\,503\,599\,627\,370\,495}{4\,503\,599\,627\,370\,496})\\
  &= 1.79769\cdot 10^{308}.
\end{align}

Since we've hinted to it already in this writing, we should probably
enumerate the special cases of the representation sooner or later.
Doing it now doesn't sound like a bad idea.
\begin{table}[h!]
  \begin{center}
    \begin{tabular}{|c|c|c|}
      \hline
      $c-b$ & $m$ & Object Represented\\
      \hline
      \hline
      0 & 0 & zeroes\\
      0 & nonzero & denormalized numbers\\
      1 to $e_{\mathrm{max}}$ & anything & normalized numbers\\
      $e_{\mathrm{max}}+1$ & 0 & infinities\\
      $e_{\mathrm{max}}+1$ & nonzero & NaNs\\
      \hline
    \end{tabular}
  \end{center}
  \caption{IEEE 754 encoding of floating-point numbers.}
  \label{special_cases}
\end{table}
The only problem with Table~\ref{special_cases} is that we haven't yet
covered what a denormalized number is.

\begin{quea}
  What is a denormalized number?
\end{quea}
\begin{ansa}
  A denormalized number is a number in the world that is
  floating-point math that fills in the gap between zero and the
  smallest (largest) positive (negative) number.  These are only
  important to consider from a numerical analysis standpoint because
  they have interesting implications when involved in arithmetic
  operations, namely division by zero and $a-b=0$ for $a\not =b$.
\end{ansa}

\section{More Floating-Point Numbers}

So, what more is there?  For starters, we can talk about arithmetic
operations like addition, subtraction, and so on and their pitfalls
but we'll probably leave this for last or near to last.

There's also the possibility for discussing conversions between
decimal representations and IEEE 754 representations, which we'll most
likely discuss next.  Furthermore, we can even get into some of the
interesting analytic topics like the spacing of IEEE 754 values on a
number line or error analysis or ordering or the like, but it's
unlikely that we'll get too far with this because of the mathematical
background of the participants in 312 and because of the fact that
this is not a numerical analysis course.

Still, there's the topic of rounding, but that also tends to involve
numerical analysis.

\subsection{Converting from Decimal to IEEE 754}

The conversion from decimal to IEEE 754 isn't exactly the most
straightforward at first, but the process pretty much goes like the
following.  Assume that we have a real number $x$ in decimal.
\begin{enumerate}
\item Convert $x$ to binary.
\item Normalize the binary representation of $x$.
\item Let $e$ be the amount of binary places shifted in the
  normalization process.  Set $c-b=e$ in~\eqref{ieee754_form}.
\item Assemble the pieces.
\item ???
\item PROFIT!
\end{enumerate}
There are exercises in the text that cover this, but just for the sake
of being at least halfway complete here, let's work out an example.
\begin{exma}
  Convert $\pi\approx 3.14159$ to IEEE 754 double-precision format.
\end{exma}
\begin{sola}
  First, we convert $3.14159$ to binary, which yields
  \begin{equation*}
    \pi_2\approx 11.001001000011111101_2.
  \end{equation*}
  We normalize $\pi_2$, which gets us
  \begin{equation*}
    \pi_2\approx 1.1001001000011111101_2\cdot 10_2^{1}.
  \end{equation*}
  This means that $e=1$.  We can now start assembling the pieces.  We
  have
  \begin{equation*}
    \pi_2\approx (-1)^0\cdot 2^{1}\cdot (1+2^{-1}+2^{-4}+2^{-7}+2^{-12}+\cdots+2^{-17}+2^{-19}).
  \end{equation*}
  This is our IEEE 754 double-precision representation for an
  approximation of $\pi$ to six significant digits.
\end{sola}

\end{document}
